%This file contains all new environments simplifying LaTeX text editing
%For each new environment clearly document its dependencies (packages) and its use
%Adhere to the ordering by category!

%----------------------------------------------------------------------------------------------
%Tables
%----------------------------------------------------------------------------------------------
\NewDocumentEnvironment{TTable}{O{1}mmmm} %Defines a basic tabular environment
{
\begin{table}[H]
\centering
\def\counter{#1}
\if\counter2
\rowcolors{#1}{gray!25}{white}
\else
\rowcolors{#1}{white}{gray!25}
\fi
\begin{tabular}{#2}
\toprule
}
{
\bottomrule
\end{tabular}
\caption[#3]{#4}
\label{#5}
\end{table}
}
%Usage:
%\begin{TTable}[<startindex for colors>]
%{<table layout>}
%{<your short caption>}
%{<your caption>}
%{<your label>}
%<your Content>
%\end{TTable}
%Dependencies: \usepackage{booktabs,xcolors,xparse}



\NewDocumentEnvironment{TTable*}{mmmm} %Defines a basic tabular environment
{
\begin{table}[H]
\centering
\begin{tabular}{#1}
\toprule
}
{
\bottomrule
\end{tabular}
\caption[#2]{#3}
\label{#4}
\end{table}
}
%Usage:
%\begin{TTable*}{<table layout>}
%{<your short caption>}
%{<your caption>}
%{<your label>}

%<your Content>

%\end{TTable*}


\NewDocumentEnvironment{TTableNonFloat*}{m} %Defines a basic tabular environment
{
\begin{center}
\begin{tabular}{#1}
\toprule
}
{
\bottomrule
\end{tabular}
\end{center}
}
%Usage:
%\begin{TTable*}{<table layout>}
%{<your short caption>}
%{<your caption>}
%{<your label>}

%<your Content>

%\end{TTable*}


\NewDocumentEnvironment{TDefinitionTable*}{} %Used to define variables in an equation
{
\rowcolors{1}{gray!25}{white}
\footnotesize
\renewcommand{\arraystretch}{1.3}
\begin{center}
\begin{tabular}{lll}
}
{
\end{tabular}
\end{center}
\normalsize
}
%Usage:
%\begin{TDefinitionTable}
%$\vect{P}$&change of linear momentum &in& $\si{\kilo\gram\meter\per\second\squared}$\\
%$m$&mass&in&$\si{\kilo\gram}$\\
%$\vec{a}$&acceleration&in&$\si{\meter\per\second\squared}$\\
%$\vec{F}$&resultant force&in&$\si{N}=\si{\kilo\gram\meter\per\second\squared}$
%\end{TDefinitionTable}

\NewDocumentEnvironment{TValueTable}{} %Used to introduce a series of parameters with certain values
{
\footnotesize
\begin{center}
\begin{tabular}{r@{ = }lr}
}
{
\end{tabular}
\end{center}
\normalsize
}
%Usage:
%\begin{TValueTable}
%$h_E$&$\SI{32}{\milli\meter}$&Extension of the brakes\\
%$h_p$&$\SI{32}{\milli\meter}$&Height of a single plate\\
%$l_B$&$\SI{80}{\milli\meter}$&Width of one brake\\
%$n_p$&$l_B/l_p=\SI{5}{}$&Number of plates\\
%$A_p$&$h_p\cdot l_p\cdot n_p=\SI{25.6}{\centi\meter\squared}$&Total area of the plates\\
%$A_c$&$\SI{26.8}{\centi\meter\squared}$&Actual area of the brake
%\end{TValueTable}

%To read a csv sheet
\newcommand{\resulttable}[5]{
\csvloop{
file=#1,
no head,
column count=#2,
before reading=\begin{table}[H]\caption{#4}\label{#5}\rowcolors{2}{white}{gray!25}\begin{longtable}{|*{#2}{l|}}\hline,
command=\csvlinetotablerow,
late after line=\if\thecsvrow #3
\\\hline\hline
\else
\\
\fi,
after reading=\hline\end{longtable}\end{table}
}
}
%Usage:
%1. Create your table on gdocs. It is possible to use latex syntax in the table!
%2. File -> Download as -> Comma seperated values (.csv)
%3. Place your file in the same folder as your .tex file
%4. Use the command as follows:
%\resulttable
%{<filename>.csv}
%{<number of columns>}
%{<number of head lines + 1>}
%{<caption of your table>}
%{<label of your table>}
%
%For example:
%
%\resulttable{Sheet1.csv}%Filename
%{5}%Number of columns
%{4}%Number of head lines + 1
%{TESTID\_EXPERIMENTID\_YYMMDD}%Experiment ID
%{label}



%----------------------------------------------------------------------------------------------
%Appendices
%----------------------------------------------------------------------------------------------
\NewDocumentEnvironment{singlePDFpage}{mmm}
{
\includepdf[pages=-,scale=.75,pagecommand={
\subsection{#1}\label{#2}
},linktodoc=true]
{appendix/#3}
}

\NewDocumentEnvironment{multiPDFpage}{mmm}
{
\includepdf[pages=1,scale=.75,pagecommand={
\subsection{#1}\label{#2}
},linktodoc=true]
{appendix/#3}}
{\includepdf[pages=2-,scale=0.85,pagecommand={},linktodoc=true]{appendix/#3}
}
